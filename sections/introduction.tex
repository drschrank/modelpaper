\section{\label{sec:intro}Introduction}
Spin-exchange optical pumping (SEOP) is a technique whereby the ensemble nuclear spin-angular momentum of certain noble gasses can be increased to of order 10\%. The technique is currently most notably used clinically and pre-clinically in lung imaging using MRI \cite{Oros2004HyperpolarizedMRI}, but it has been also been used in the NMR characterization of porous media \cite{Terskikh2002AMaterials} and protein dynamics \cite{Schroder2013XenonAlert}. 

The the physics of SEOP are described comprehensively in other places \cite{Walker2011, Appelt1998Theory129Xe}. Briefly, the technique involves two steps: optical pumping of an alkali metal vapor and spin-exchange from the alkali metal vapor to a noble gas nuclei. In the first step, optical pumping, a beam of circularly polarized light is directed to on a transparent cell containing a macroscopic amount of alkali metal. The cell is heated, usually to between 100-200 $^{\circ}$C to vaporize some amount alkali metal. The laser interacts with the metal vapor to create close to 100\% spin polarization of the alkali vapor.

In the second step, spin-exchange, the alkali vapor transfers spin-angular momentum to a noble gas nuclei. A gas mixture is introduced into the cell containing a noble gas and some other inert gasses, and through collisional interactions, the metal vapor transfers its spin-polarization to the noble gas. The alkali metal vapor atoms becomes depolarized in this interaction, but because of optical pumping, the alkali atoms are quickly repolarized.

One popular method of SEOP involves the hyperpolarization of xenon-129 ($^{129}$Xe) using rubidium (Rb) vapor. Hyperpolarized xenon-129 (HP $^{129}$Xe) gas is typically produced in a continuous manner using a flow-through polarizer \cite{Driehuys1996High-volume129Xe, Ruset2006Optical129Xe, Schrank2009a}. A flow-through polarizer operates by flowing a $^{129}$Xe gas mixture through an optical pumping cell containing Rb vapor. The $^{129}$Xe spin-exchange interaction occurs on a short enough timescale that considerable $^{129}$Xe polarizations can be achieved during the short transit through the optical pumping cell.

With increased interested in this technique for medical imaging, there is a need to develop technology which can produce highly-volume, high-polarization $^{129}$Xe on short timescales. Recently, Ref. \cite{Freeman2014} noted that some styles of flow-through polarizers were not producing HP $^{129}$Xe with the polarization that theoretical models predicted. 

Currently, there are no SEOP models that attempt to account for the full three-dimensional flow-dynamics of optical pumping cell geometries. Computational models for SEOP can be broken into two groups: finite difference models and finite element models (FEM). The finite difference models appear to have been the first in use, but detailed descriptions of the models do not appear extensively in the literature. Models of this type approximate flow through the cell by either a one dimensional plug flow model (such as those used in Ref. \cite{Freeman2014}) or two dimensional laminar flow (such as Ref. \cite{Ruset2005}). 

The first computational model of spin-exchange optical pumping to be extensively described in the literature was done so by Ref. \cite{Fink2005}. It originally described a model approximating an optical pumping cell geometry with only a half-cylinder. The model included the effects of laser heating, fluid dynamics, and heat transfer. The original model only investigated a static optical pumping cell in which the gas was not flowing. This was later extended to analyize the two geometries with gas flow: a half cylinder and flowing through a simplified geometry of the optical pumping cell used by Ref.\cite{Ruset2006Optical129Xe}.

Recently, Ref. \cite{Burant2018CHARACTERIZINGCONTRAST} has described an FEM model fluid model of an optical pumping cell. However, it does not appear the model included SEOP and only included fluid flow, heat transfer, and diffusion of Rb metal vapor. 

Here, we present an open-source FEM code that attempts to a model the dynamics of SEOP using the full three dimensional geometry of a common optical pumping cell design. The model incorporates fluid flow, Rb vapor diffusion, thermal transfer, laser absorption, and $^{129}$Xe polarization. We also present some initial, qualitative predictions of the model. 