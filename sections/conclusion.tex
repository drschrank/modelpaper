\section{Conclusion and Future Work\label{sec:conclusion}}
A new open-source, FEM based SEOP model has been coded, tested against an existing model, and used to attempt to understand the dynamics in complex geometries. The model provides predictions of HP $^{129}$Xe polarizations that are comparable to existing, accepted models when the conditions are limited to the scope of those existing models. The new FEM model provides the ability to visualize important SEOP phenomena such as laser heating, Rb vapor distribution, and gas flow. The model can compute solutions for complicated geometries including current designs for optical pumping cells.

The FEM model described here can be improved in many ways. Most immediately, the model lacks multi-threading capabilities. This limits the speed at which solutions can be computed. Most results presented in this paper took $\approx$300-700 hours to compute because the process was single-threaded. This limits transient studies to simulate only the first several hundred seconds after initialization, and important, long-term behavior cannot be thoroughly investigated. Multi-threading capabilities with appropriate increases in other computational resources could potential decrease the amount of time required to calculate a single time step and open up the possibility of exploring the long-term behavior of these systems.

The transient FEM model is also, currently, not stable under important boundary conditions. Almost all the simulations of complex geometries terminated prematurely due to a failure of one of the linear solvers to converge at some time step before what was prescribed. Although it is possible to restart the computation, this seems to create artifacts in the simulated dynamics. It takes several time steps before the dynamical behavior returns to its pre-restart state. 

Those two improvements would allow for the model to investigate a steady-state HP $^{129}$Xe polarization at the outlet. The current study was unable to achieve that because steady-state solutions were not observed.

Finally, better approximations to spin-exchange parameters and other model parameters may increase the fidelity of the model's predictions. However, the major source of error in the model is likely due to the choice of Rb source distribution. Because all the parameters are correct to an order of magnitude, it is unlikely substantially gains will be realized by pursuing more accurate approximations to these parameters unless it can be shown that the current estimates of the parameters are incorrect by large margins.

The initial simulations computed using this FEM model have revealed qualitative insights about current designs for optical pumping cells. In particular, the model indicates that cell performance is strongly linked to Rb source distribution. Rb sources that are near the ``top'' of the cell or closest to the where the pump laser beam enters the cell can contribute more strongly to the overall Rb vapor number density because of heating due to the laser and gas. Even a Rb source that is nominally upstream of the cell body can diffuse back into the body and strongly effect the dynamics of optical pumping. 

Although in this model, Rb sthe ource distribution is static, there is no reason to believe this is the case in actual SEOP systems. Rb metal will redistribute itself into different configurations during the course of the life of the cell, and these configurations may not be optimum for optical pumping. In addition to other mechanisms, this reconfiguration, particularly the accumulation of Rb in the outlet tube, may contribute to the aging and eventual failure of optical pumping cells

In addition to affecting the Rb vapor number density in the optical pumping cell body, Rb metal films in the outlet of the cell may strongly effect the overall achievable polarization of HP $^{129}$Xe in the system. This is because gasses, heated by the laser in the cell body, flow over and heat Rb films deposited in the outlet. This may cause high densities of Rb vapor to form in the outlet outside of the region illuminated by the pump laser. These Rb vapors would be completely depolarized and may contribute to rapid depolarization of HP $^{129}$Xe exiting the optical pumping cell body.

The conclusions listed above are tentative as a configuration of Rb metal in the optical pumping cell that yields a steady-state solution has yet to be definitively identified. However, laser heating of the gas in SEOP systems is already a well established phenomena\cite{Ruset2005}, and Ref. \cite{Fink2005} established theoretically the the phenomena of convective transfer of heat against a gravitational field is a well understood principle. The model provides insight into understanding the poor performance of current optical pumping systems without the need to appeal to cell contamination or other mechanisms. As improvements are made to the model, it may eventually become necessary to appeal to mechanisms outside of what is described in the model. However, before accepting these explanations, every effort should first be made to understand the behaviour of optical pumping systems in terms of known physical principles. 