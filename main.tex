%Start of file apssamp.tex ******
%
%   This file is part of the APS files in the REVTeX 4.1 distribution.
%   Version 4.1r of REVTeX, August 2010
%
%   Copyright (c) 2009, 2010 The American Physical Society.
%
%   See the REVTeX 4 README file for restrictions and more information.
%
% TeX'ing this file requires that you have AMS-LaTeX 2.0 installed
% as well as the rest of the prerequisites for REVTeX 4.1
%
% See the REVTeX 4 README file
% It also requires running BibTeX. The commands are as follows:
%
%  1)  latex apssamp.tex
%  2)  bibtex apssamp
%  3)  latex apssamp.tex
%  4)  latex apssamp.tex
%
\documentclass[%
 %preprint,
%superscriptaddress,
%groupedaddress,
%unsortedaddress,
%runinaddress,
%frontmatterverbose, 
reprint,
%showpacs,preprintnumbers,
%nofootinbib,
%nobibnotes,
%bibnotes,
 amsmath,amssymb,
 aps,
pra,
%prb,
%rmp,
%prstab,
%prstper,
floatfix,
]{revtex4-1}

\usepackage{graphicx}% Include figure files
\usepackage[caption=false]{subfig}%For subfigures
\usepackage{dcolumn}% Align table columns on decimal point
\usepackage{bm}% bold math
\usepackage{hyperref}% add hypertext capabilities
\usepackage[mathlines]{lineno}% Enable numbering of text and display math
\usepackage{import}
\usepackage{longtable}
\usepackage[table]{xcolor}
\usepackage{array}
\newcolumntype{?}{!{\vrule width 1pt}} %for larger lines in tables
\newcommand{\rulesep}{\unskip\ \vrule\ }%for vertical rule in subfloats
%\linenumbers\relax % Commence numbering lines

%\usepackage[showframe,%Uncomment any one of the following lines to test 
%%scale=0.7, marginratio={1:1, 2:3}, ignoreall,% default settings
%%text={7in,10in},centering,
%%margin=1.5in,
%%total={6.5in,8.75in}, top=1.2in, left=0.9in, includefoot,
%%height=10in,a5paper,hmargin={3cm,0.8in},
%]{geometry}

\begin{document}

\preprint{APS/123-QED}

\title{An Open-Source Finite Element Model for Spin-Exchange Optical Pumping In Three Dimensional Geometries}% Force line breaks with \\
%\thanks{A footnote to the article title}%

\author{G.M. Schrank}
 \affiliation{Dept. of Radiology, Duke University Medical Center, 311 Research Dr. 27707, USA}%Lines break automatically or can be forced with \\
%\author{}%
% \email{Second.Author@institution.edu}
%\affiliation{%
% Authors' institution and/or address\\
% This line break forced with \textbackslash\textbackslash
%}%

%\collaboration{MUSO Collaboration}%\noaffiliation

%\author{Charlie Author}
% \homepage{http://www.Second.institution.edu/~Charlie.Author}
%\affiliation{
 %Second institution and/or address\\
 %This line break forced% with \\
%}
%\affiliation{
% Third institution, the second for Charlie Author
%}%
%\author{Delta Author}
%\affiliation{%
% Authors' institution and/or address\\
% This line break forced with \textbackslash\textbackslash
%}%

%\collaboration{CLEO Collaboration}%\noaffiliation

\date{\today}% It is always \today, today,
             %  but any date may be explicitly specified

\begin{abstract}
We present a new spin-exchange optical pumping model that utilizes the open-source finite element code, ElmerFEM-CSC. The model simulates the full, three-dimensional geometry of optical pumping cells and incorporates fluid flow, thermal transfer, alkali diffusion, laser absorption, and spin-exchange. The new model has been compared with a simpler computational model, and the results are comparable within the limitations of the simpler model. Initial simulations using optical pumping cell geometries reveal that the performance of some common optical pumping systems are critically dependent on the distribution of alkali metal vapor sources in the cell, and that alkali metal vapor sources in outlet tubing may contribute to rapid depolarization in hyperpolarized gasses. 
%\begin{description}
%\item[Usage]
%Secondary publications and information retrieval purposes.
%\item[PACS numbers]
%May be entered using the \verb+\pacs{#1}+ command.
%\item[Structure]
%You may use the \texttt{description} environment to structure your abstract;
%use the optional argument of the \verb+\item+ command to give the category of each item. 
%\end{description}
\end{abstract}

\pacs{Valid PACS appear here}% PACS, the Physics and Astronomy
                             % Classification Scheme.
%\keywords{Suggested keywords}%Use showkeys class option if keyword
                              %display desired
\maketitle

%\tableofcontents

\import{sections/}{introduction.tex}
\import{sections/}{modeldescription.tex}
\import{sections/}{verification.tex}
\import{sections/}{results.tex}
\import{sections/}{discussion.tex}
\import{sections/}{conclusion.tex}
\begin{acknowledgments}
The author gratefully acknowledges Dr. Anderson and Mr. Cook for useful conversations regarding the coding and physics of the simulations. Also, much of the computation time on Amazon's AWS Cloud Computing Service was supported by backers of the project from Experiments.com. The author gratefully acknowledges the financial support of all the backers of the project.
\end{acknowledgments}

\appendix
\import{sections/}{modeldetails.tex}
\import{sections/}{diffusionderivation.tex}

\bibliographystyle{unsrt}
\bibliography{references}

\end{document}
%
% ****** End of file apssamp.tex ******